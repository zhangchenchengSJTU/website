\documentclass{article}
\usepackage[a4paper, left=0.5in, right=0.5in, top=0.5in, bottom=0.5in]{geometry} % Page margins
\usepackage{graphicx} % Required for inserting images
\usepackage{amsthm, amsmath, amssymb, mathtools}  % Required for math typesetting
% \usepackage{quiver} % Required for commutative diagrams
\usepackage[all, cmtip, 2cell]{xy}
\UseTwocells

\usepackage[colorlinks]{hyperref} % For hyperlinks in the PDF
\usepackage[nameinlink]{cleveref} % For referencing multiple labels

\title{Homological Theories for Extriangulated Categories}
\author{Tansing Tiunn}
\date{\today}


% Define theorem styles and environments
\theoremstyle{plain}  % for theorems, etc. (italic body)
\newtheorem{theorem}{Theorem}[section]
\newtheorem{lemma}[theorem]{Lemma}
\newtheorem{proposition}[theorem]{Proposition}
\newtheorem{corollary}[theorem]{Corollary}
\newtheorem{conjecture}[theorem]{Conjecture}

\theoremstyle{definition}  % for definitions, examples (upright body)
\newtheorem{definition}[theorem]{Definition}
\newtheorem{example}[theorem]{Example}
\newtheorem{problem}[theorem]{Problem}

\newtheorem*{notation}{Notation}

\theoremstyle{remark}  % for remarks etc. (upright, less emphasis)
\newtheorem*{remark}{Remark}
\newtheorem*{note}{Note}


\begin{document}

\maketitle

\section{Homotopy for Finitely Presented Functors}

\subsection{Finitely Presented Functors}

\begin{notation}
    Throughout, all categories and functors are assumed to be additive. We write $\mathrm{Fun}(\mathcal{A}, \mathcal{B})$ for the category of additive functors from $\mathcal{A}$ to $\mathcal{B}$.
\end{notation}

\begin{definition}[Finitely presented functors]
    Let $\mathcal{A}$ be an additive category. A functor $F:\mathcal{A}^{\mathrm{op}} \to \mathbf{Ab}$ is called finitely presented if there exists an exact sequence of functors
    \begin{equation}\label{eq:fp_functor}
        \mathrm{Hom}_{\mathcal{A}}(-, X) \to \mathrm{Hom}_{\mathcal{A}}(-,Y) \to F \to 0
    \end{equation}
    for some objects $X,Y \in \mathcal{A}$.
\end{definition}

\begin{remark}
    By Yoneda's lemma, the natural transformations $\mathrm{Hom}_{\mathcal{A}}(-, X) \to \mathrm{Hom}_{\mathcal{A}}(-,Y)$ correspond to morphisms $X \to Y$ in $\mathcal{A}$.
\end{remark}

\begin{notation}
We denote the category of additive functors from $\mathcal{A}^{\mathrm{op}}$ to $\mathbf{Ab}$ by $\mathrm{Mod}_\mathcal{A}$, and the category of finitely presented functors by $\mathrm{mod}_\mathcal{A}$.
\end{notation}

\begin{notation}
    We write for simplicity $\mathrm{Hom}_{\mathcal{A}}(-,?)$ as $(-, ?)_\mathcal{A}$, or $(-, ?)$ when there is no ambiguity.
\end{notation}

We show some closure properties of $\mathrm{mod}_\mathcal{A} \rightarrowtail \mathrm{Mod}_\mathcal{A}$.

\begin{proposition}\label{prop:fp_cok}
    $\mathrm{mod}_\mathcal{A}$ is closed under cokernels.
    \begin{proof}
        Let $F, G \in \mathrm{mod}_\mathcal{A}$ and $h: F \to G$ be a natural transformation. We want to show that $\mathrm{Coker}(h) \in \mathrm{mod}_\mathcal{A}$. Since $F$ and $G$ are finitely presented, there exists a commutative diagram of right exact sequences
        \begin{equation}
            \xymatrix{
(-, X^{-1}) \ar@{->}[r]^{{(-,f)}} \ar@{-->}[d]^{{(-, \varphi^{-1})}} & (-, X^0) \ar@{->>}[r]^{p} \ar@{-->}[d]^{{(-, \varphi^0)}} & F(-) \ar@{->}[d]^{h} \\
(-, Y^{-1}) \ar@{->}[r]^{{(-,g)}} & (-, Y^0) \ar@{->>}[r]^{q} & G(-)
            }.
        \end{equation}
        The morphisms $(-, \varphi^\bullet)$ are induced by projective objects. We take the mapping cone of the left square to obtain the following commutative diagram
        \begin{equation}\label{eq:fp_cok_diag}
\xymatrix{
(-, X^{-1}) \ar@{->}[rr]^{{(-,f)}} \ar@{->}[d]^{{(-, \varphi^{-1})}} &  & (-, X^0) \ar@{->}[d]^{{(-, \varphi^{0})}} & \mathfrak X^\bullet \ar@{->}[d]^{\Phi ^\bullet} \\
(-, Y^{-1}) \ar@{->}[rr]^{{(-,g)}} \ar@{->}[d]^{{(-, \binom 10)}} &  & (-, Y^0) \ar@{->}[d]^{\mathrm{id}} & \mathfrak Y^\bullet \ar@{->}[d] \\
(-, Y^{-1}\oplus X^0) \ar@{->}[rr]^{{(-, (g,-\varphi^0))}} &  & (-, Y^0) & \mathrm{Cone}(\Phi^\bullet)
}.
        \end{equation}
        By long exact sequence, we obtain
        \begin{equation}
            \operatorname{cok} (h) \simeq \operatorname{cok}\left[ H^0 (\mathfrak X^\bullet) \to H^0 (\mathfrak Y^\bullet)\right] \simeq \operatorname{cok}\left[ (-, (g, - \varphi ^0)) \right].
        \end{equation}
    \end{proof}
\end{proposition}

\begin{proposition}\label{prop:fp_ext}
    $\mathrm{mod}_\mathcal{A}$ is closed under extensions. That is, given a short exact sequence $0 \to F \to G \to H \to 0$ in $\mathrm{Mod}_\mathcal{A}$, if $F, H \in \mathrm{mod}_\mathcal{A}$, then $G \in \mathrm{mod}_\mathcal{A}$.
    \begin{proof}
        Let $\mathfrak P^\bullet$ and $\mathfrak R^\bullet$ be presentations of $F$ and $H$, respectively. By horseshoe lemma, there exists a presentation $\mathfrak Q^\bullet$ of $G$ which fits into a degreewise split short exact sequence
        \begin{equation}
            0 \to \mathfrak P^\bullet \to \mathfrak Q^\bullet \to \mathfrak R^\bullet \to 0.
        \end{equation}
        Hence, $\mathfrak Q^\bullet$ has representable components. This completes the proof.
    \end{proof}
\end{proposition}

\begin{remark}\label{rem:fp_colim}
    The finite colimits (equivalently characterised by biproducts and cokernels) in $\mathrm{mod}_\mathcal{A}$ are the same as those in $\mathrm{Mod}_\mathcal{A}$. In other words, finite colimits of finitely presented functors are computed pointwise.
    
    Not all colimits in $\mathrm{mod}_\mathcal{A}$ are computed pointwise. For instance, let $\mathcal{A}$ be any category that admits arbitrary coproducts. We claim that $\coprod_{i \in I}(-, X_i) \simeq (-, \coprod _{i\in I}X_i)$ in $\mathrm{mod}_\mathcal{A}$. All functors in $\mathrm{mod}_\mathcal{A}$ preserves products, as products are exact in $\mathbf{Ab}$. By Yoneda lemma,
    \begin{equation}
        \left[(-, \coprod _{i\in I}X_i), F(-)\right] \simeq F\left(\coprod _{i\in I}X_i\right) \simeq \prod _{i \in I} F(X_i) \simeq \prod _{i\in I}\left[(-, X_i), F(-)\right]\simeq \left[\coprod _{i\in I}(-, X_i), F(-)\right].
    \end{equation}
\end{remark}

We show that representable functor are projective, making \cref{eq:fp_functor} a projective presentation.

\begin{lemma}\label{lem:rep_proj}
    Representable functors are projective in both $\mathrm{Mod}_\mathcal{A}$ and $\mathrm{mod}_\mathcal{A}$.
    \begin{proof}
        Since cokernels in $\mathrm{mod}_\mathcal{A}$ is computed in $\mathrm{Mod}_\mathcal{A}$ (\cref{prop:fp_cok}), it suffices to show that representable functors are projective in $\mathrm{Mod}_\mathcal{A}$. Let $P = (-, X)$ be a representable functor, and let $\theta : F \twoheadrightarrow G$ be an epimorphism in $\mathrm{Mod}_\mathcal{A}$. By Yoneda lemma,
        \begin{equation}
            \left[(-, X), \theta \right] \simeq [\theta _X : F(X) \twoheadrightarrow G(X)]
        \end{equation}
        is epic. Hence $(-, X)$ is projective.
    \end{proof}
\end{lemma}

\begin{remark}
    The converse of \cref{lem:rep_proj} is not ture. For instance, suppose that there is an idempotent $e \in (X,X)$ such that $\operatorname{cok} e$ is non-representable. Then $\operatorname{cok}(-, e)$ is not representable, but it is projective as it is a direct summand of the projective object $(-, X)$.
\end{remark}

\subsection{When \texorpdfstring{$\mathrm{mod}_\mathcal{A}$}{} is Abelian}

We learn by \cref{prop:fp_ext} and \cref{prop:fp_cok} that $\mathrm{mod}_\mathcal{A}$ is an full subcategory of the Abelian category $\mathrm{Mod}_\mathcal{A}$, which is closed under extensions and cokernels. However, $\mathrm{mod}_\mathcal{A}$ is lack of kernels in general.

\begin{definition}[Weak kernel]\label{def:weak_kernel}
    Say $K \xrightarrow{i} X$ is a weak kernel of $X \xrightarrow f Y$ if for any $j: W \to X$ such that $fj = 0$, there exists a (not necessarily unique) morphism $k: W \to K$ such that $jk = i$.
\end{definition}

\begin{lemma}\label{lem:weak_kernel}
    $K \xrightarrow{i} X$ is a weak kernel of $X \xrightarrow f Y$, if and only if $(-, K) \xrightarrow{(-, i)} (-, X) \xrightarrow{(-, f)} (-, Y)$ is exact at $(-, X)$, computed pointwise.
    \begin{proof}
        \cref{def:weak_kernel} means precisely $\ker (-, f) = \operatorname{im} (-, i)$.
    \end{proof}
\end{lemma}

\begin{example}
    Kernels are weak kernels. For any distinguished triangle $X \xrightarrow u Y \xrightarrow v Z \xrightarrow w \Sigma X$ in a (pre-)triangulated category, $X \xrightarrow u Y$ is a weak kernel of $Y \xrightarrow v Z$.
\end{example}

\begin{remark}
    We can define weak cokernels dually. However, $f$ is not always a weak cokernel of $i$ in \cref{def:weak_kernel}.
\end{remark}

\begin{example}\label{ex:fp_proj_res}
    Suppose $\mathcal{A}$ has weak kernels. Any $F \in \mathrm{mod}_\mathcal{A}$ admits a projective resolution of length $\infty$:
    \begin{equation}
        \cdots \xrightarrow{(-, f^{-2})} (-, X^{-2}) \xrightarrow{(-, f^{-1})} (-, X^{-1}) \xrightarrow{(-, f^0)} (-, X^0) \xrightarrow p F \to 0,
    \end{equation}
    wherein $f^{-(k+1)}$ is a weak kernel of $f^{-k}$ for each $k \geq 0$.
\end{example}

\begin{theorem}\label{thm:fp_abelian}
    $\mathrm{mod}_\mathcal{A}$ has kernels (computed pointwise in $\mathrm{Mod}_\mathcal{A}$) if and only if $\mathcal{A}$ has weak kernels.
    \begin{proof}
        ($\Rightarrow$). We show that any $f : X \to Y$ has kernels. Set $F = \operatorname{cok} (-, f)$. By assumption, there are $H, K \in \mathrm{mod}_\mathcal{A}$ while $(i', p')$ and $(i, p)$ are kernel-cokernel pairs in $\mathrm{Mod}_\mathcal{A}$:
        \begin{equation}
            \xymatrix{
(-, M) \ar@{->>}[rd]^{p''} \ar@{-->}[rr]^{i'\circ p''} &  & (-, X) \ar@{->}[rr]^{{(-,f)}} \ar@{->>}[rd]^{p'} &  & (-, Y) \ar@{->>}[r]^{p} & F \\
 & H \ \ar@{>->}[ru]^{i'} &  & K \ \ar@{>->}[ru]^{i} &  & 
            }.
        \end{equation}
        Clearly, $\operatorname{im}(i' \circ p'') \simeq H \simeq \ker p' = \ker (-,f)$. By Yoneda lemma, $i' \circ p'' = (-, l)$ for some $l : M \to X$. Hence, $l$ is a weak kernel of $f$ (\cref{lem:weak_kernel}).

        ($\Leftarrow$). When $\mathrm{mod}_\mathcal{C}$ has weak kernels, we extend \cref{eq:fp_cok_diag} to 3-term presentations:
        \begin{equation}
\xymatrix{
(-, X^{-2}) \ar@{-->}[rr]^{{(-,a)}} \ar@{->}[d]^{{(-, \varphi ^{-2})}} &  & (-, X^{-1}) \ar@{->>}[rr]^{{(-,f)}} \ar@{->}[d]^{{(-, \varphi^{-1})}} &  & (-, X^0) \ar@{->}[d]^{{(-, \varphi^0)}} & \mathfrak X^\bullet \ar@{->}[d]^{\Phi ^\bullet} \\
(-, Y^{-2}) \ar@{-->}[rr]^{{(-, b)}} \ar@{>->}[d]^{{(-, \binom 10)}} &  & (-, Y^{-1}) \ar@{->>}[rr]^{{(-,g)}} \ar@{>->}[d]^{{(-, \binom 10)}} &  & (-, Y^0) \ar@{=}[d] & \mathfrak Y^\bullet \ar@{->}[d] \\
(-, Y^{-2}\oplus X^{-1}) \ar@{-->}[rr]^{{(-, \binom{b,-\varphi ^{-1}}{0, -f})}} &  & (-, Y^{-1}\oplus X^0) \ar@{->>}[rr]^{{(-, (g,-\varphi^0))}} &  & (-, Y^0) & \mathrm{Cone}(\Phi^\bullet)
}.
        \end{equation}
        By long exact sequence, we obtain 
        \begin{equation}
\xymatrix{
H^{-1}\mathfrak Y^\bullet \ar@{->}[r] \ar@{=}[d] & H^{-1}\mathrm{Cone}(\Phi^\bullet) \ar@{->}[r] \ar@{->>}[d] & H^0\mathfrak X^\bullet \ar@{=}[d] \ar@{->}[r] & H^0\mathfrak Y^\bullet \ar@{=}[d] \ar@{->}[r] & H^0\mathrm{Cone}(\Phi^\bullet) \ar@{=}[d] \ar@{->}[r] & H^1\mathfrak X^\bullet \ar@{=}[d] \\
0 \ar@{->}[r] & \ker h \ \ \ar@{>->}[r] & F \ar@{->}[r]^{h} & G \ar@{->>}[r] & \operatorname {coker}h \ar@{->}[r] & 0
}.
        \end{equation}
        It suffices to show that $H^{-1}\mathrm{Cone}(\Phi ^\bullet) \simeq \frac{\ker (-, (g,-\varphi^0))}{\operatorname{im}(-, \binom{b,-\varphi ^{-1}}{0, -f})} \in \mathrm{mod}_\mathcal{C}$. By \cref{prop:fp_cok}, we show $\ker (-, (g,-\varphi^0))\in \mathrm{mod}_\mathcal{C}$ and $\operatorname{im}(-, \binom{b,-\varphi ^{-1}}{0, -f}) \in \mathrm{mod}_\mathcal{C}$. Note that
        \begin{equation}
            \operatorname{im}(-, \binom{b,-\varphi ^{-1}}{0, -f}) \simeq \operatorname{cok} \left[(-, Y^{-3}\oplus X^{-2}) \to (-, Y^{-2}\oplus X^{-1})\right] \in \mathrm{mod}_\mathcal{C}.
        \end{equation}
        A computation of kernel-cokernel sequence shows the short exact sequence
        \begin{equation}
            0 \to \ker (-, g) \to \ker (-, (g,-\varphi^0)) \to \operatorname{cok} (-, \binom 1 0) \to 0.
        \end{equation}
        By \cref{prop:fp_ext}, we see $\ker (-, (g,-\varphi^0)) \in \mathrm{mod}_\mathcal{C}$.
    \end{proof}
\end{theorem}

\begin{corollary}\label{cor:fp_abelian}
    Under the assumption of \cref{thm:fp_abelian}, there is a canonical isomorphism $\operatorname{cok} \ker \simeq \ker \operatorname{cok}$ in $\mathrm{mod}_\mathcal{A}$ which is computed pointwise. If any only if $\mathcal{A}$ has weak kernels, $\mathrm{mod}_\mathcal{A}$ is an Abelian subcategory of $\mathrm{Mod}_\mathcal{A}$ whose finite limits and colimits are computed pointwise.
\end{corollary}

We slightly enhance \cref{thm:fp_abelian} to get rid of the pointwise condition.

\begin{lemma}\label{lem:fp_kernel}
    When $\mathrm{mod}_\mathcal{A}$ has kernels, then the kernel must be computed pointwise.
    \begin{proof}
        It suffices to show that $\mathcal{A}$ has weak kernels (\cref{thm:fp_abelian}). Let $f: X \to Y$ be any morphism in $\mathcal{A}$. By assumption, we take presentation of $\ker (-, f)$ and obtain 
        \begin{equation}
            \xymatrix{
(-, N) \ar@{->}[rr]^{{(-, g)}} &  & (-, M) \ar@{->>}[rd]^{p} \ar@{-->}[rr]^{{(-, \varphi )}} &  & (-, X) \ar@{->}[rr]^{{(-,f)}} &  & (-, Y) \\
 &  &  & K \ \ar@{>->}[ru]^{i} &  &  & 
            }.
        \end{equation}
        By Yoneda lemma, $i \circ p$ takes the form $(-, \varphi)$. We claim that $\varphi$ is a weak kernel of $f$. Indeed, any morphism $j: W \to X$ corresponds to a natural transformation $(-, j) : (-, W) \to (-, X)$. When $f \circ j = 0$, we see $(-, j)$ factors through $i$. By lifting property projective objects, $(-, j)$ factors through $(-, \varphi)$. By Yoneda lemma again, $j$ factors through $\varphi$. This completes the proof.
    \end{proof}
\end{lemma}

\begin{remark}
    We mention that \cref{lem:fp_kernel} is a non-trivial but a satisfying coincidence. The general (co)limits in $\mathrm{mod}_\mathcal{A}$ might not be computed pointwisely, see \cref{rem:fp_colim}.
\end{remark}

\begin{theorem}
    $\mathrm{mod}_\mathcal{A}$ has kernel iff $\mathcal{A}$ has weak kernels, iff $\mathrm{mod}_\mathcal{A}$ has kernels computed pointwise, iff $\mathcal{A}$ is Abelian.
    \begin{proof}
        By \cref{thm:fp_abelian}, \cref{cor:fp_abelian}, and \cref{lem:fp_kernel}.
    \end{proof}
\end{theorem}

\subsection{Homological Universal Properties}

\begin{definition}[Yoneda embedding]
    The Yoneda embedding is defined to be the fully faithful functor
    \begin{equation}
        \mathbb{Y} : \mathcal{A} \to \mathrm{mod}_\mathcal{A}, \quad X \mapsto (-, X).
    \end{equation}
\end{definition}

\begin{remark}\label{rem:yoneda_exact}
    By universal property of $\mathrm{Hom}$, the Yoneda embedding preserves all limits, and thus is left exact.
\end{remark}

We with \cref{rem:yoneda_exact} to be a right adjoint. 

\begin{proposition}\label{prop:fp_left_adj}
    The Yoneda embedding $\mathbb{Y} : \mathcal{A} \to \mathrm{mod}_\mathcal{A}$ admits a left adjoint iff $\mathcal{A}$ has cokernels.
    \begin{proof}
    The left adjoit exists iff $[F, \mathbb Y(-)]_{\mathrm {mod}_\mathcal{A}}$ is representable for any $F \in \mathrm{mod}_\mathcal{A}$. We take arbitrary $F = \operatorname{cok}(-, f)$. By Yoneda lemma,
    \begin{equation}
        [F, \mathbb Y(-)]_{\mathrm {mod}_\mathcal{A}} \simeq \ker [\mathbb Y(f), \mathbb Y(-)]_{\mathrm {mod}_\mathcal{A}} \simeq \ker (f, -)_\mathcal{A}.
    \end{equation}
    The functor is representable iff $f$ has cokernel. 
    \end{proof}
\end{proposition}

\begin{corollary}
    When $\mathcal{A}$ has cokernel, the assignment
    \begin{equation}
        \mathrm{mod}_\mathcal{A} \to \mathcal{A},\quad \operatorname{cok} (-, f) \mapsto \operatorname{cok} f
    \end{equation}
    is a well-defined functor preserving all colimits.
\end{corollary}

\begin{theorem}\label{thm:fp_univ_prop}
    Let $\mathcal{A}$ and $\mathcal{B}$ be an additive category. Suppose that $\mathcal{B}$ has cokernels. The following functor categories are equivalent:
    \begin{equation}\label{eq:fp_univ_prop}
        (\mathbb Y_\mathcal{A})^\ast : \mathrm{REx}(\mathrm{mod}_\mathcal{A}, \mathcal{B}) \to \mathrm{Fun}(\mathcal{A}, \mathcal{B}), \quad T \mapsto T \circ \mathbb Y_\mathcal{A}.
    \end{equation}
    Here $\mathrm{REx} \subseteq \mathrm{Fun}$ consists of all right exact (i.e., cokernel-preserving) functors.
    \begin{proof}
        We take the adjunction $L_\mathcal{B} \dashv \mathbb Y_\mathcal{B}$. Any functor $F: \mathcal{A} \to \mathcal{B}$ extends to a right exact functor $\tilde F : \mathrm{mod}_\mathcal{A} \to \mathrm{mod}_\mathcal{B}$. 
        \begin{equation}
            \xymatrix{
\mathcal A \ar@{->}[rr]^{F} \ar@{->}[d]^{\mathbb Y_{\mathcal A}} &  & \mathcal B \ar@/^/@{->}[d]^{\mathbb Y_{\mathcal B}} \\
\mathrm{mod}_{\mathcal A} \ar@{->}[rr]^{\widetilde F} &  & \mathrm{mod}_{\mathcal B} \ar@/^/@{-->}[u]^{L_{\mathcal B}}
            }.
        \end{equation}
        Composing with the left adjoint, we obtain a right exact functor
        \begin{equation}
            L_\mathcal{B} \circ \tilde F : \mathrm{mod}_\mathcal{A} \to \mathcal{B}.
        \end{equation}
        We show $F \mapsto L_\mathcal{B} \circ \tilde F$ is a inverse of $(\mathbb Y_\mathcal{A})^\ast$. Over the level of objects,
        \begin{enumerate}
            \item $(\mathbb Y_\mathcal{A})^\ast (L_\mathcal{B} \circ \tilde F) = (L_\mathcal{B} \circ \tilde F) \circ \mathbb Y_\mathcal{A} = L_\mathcal{B} \circ (\mathbb Y_\mathcal{B} \circ F) \simeq \mathrm{id}_\mathcal{B} \circ F$, and
            \item $L_\mathcal{B} \circ \widetilde{\mathbb Y_\mathcal{A}^\ast (G)} \simeq L_\mathcal{B} \circ (\mathbb Y_\mathcal{B} \circ G) \simeq \mathrm{id}_\mathcal{B} \circ G$.
        \end{enumerate}
        For natural transformations, we show the correspondence:
        \begin{equation}
\xymatrix@C+2pc{
\mathrm{mod}_\mathcal{A} \rtwocell^{F}_{G}{\;\;\;\theta} & \mathcal{B}
} \leftrightarrow \{\theta_F\}_{F \in \mathrm{mod}_\mathcal{A}} \leftrightarrows \{\theta_{(-, X)}\}_{X \in \mathcal{A}} \leftrightarrow \xymatrix@C+2pc{
\mathcal{A} \rtwocell^{F\circ\mathbb Y_\mathcal{A}}_{G\circ\mathbb Y_\mathcal{A}}{\ \ \ \ \theta \mathbb Y_\mathcal{A}} & \mathcal{B}
}.
        \end{equation}
        Here $\{\theta_F\}$ determines $\{\theta_{(-, X)}\}$ by restriction, and $\{\theta_{(-, X)}\}$ determines $\{\theta_F\}$ by universal property of cokernels. 
    \end{proof}
\end{theorem}

\begin{theorem}\label{thm:fp_exact}
    Suppose in \cref{thm:fp_univ_prop} that $\mathcal{A}$ has weak kernels. Then the equivalence restricts to
    \begin{equation}
        (\mathbb Y_\mathcal{A})^\ast : \mathrm{Ex}(\mathrm{mod}_\mathcal{A}, \mathcal{B}) \to \mathrm{Fun}_{w}(\mathcal{A}, \mathcal{B}).
    \end{equation}
    Here $\mathrm{Ex} \subseteq \mathrm{REx}$ consists of all exact functors, and $\mathrm{Fun}_w \subseteq \mathrm{Fun}$ consists of all weak kernel-preserving functors.
    \begin{proof}
        By \cref{thm:fp_univ_prop}, it suffices to verify that $(\mathbb{Y}_\mathcal{A})^\ast$ is essentially surjective restricted to the subcategories. Clearly, $(\mathbb Y_\mathcal{A})^\ast$ sends exact functors to those in $\mathrm{Fun}_w (\mathcal{A}, \mathcal{B})$. Conversely, we show for any $F \in \mathrm{Fun}_w (\mathcal{A}, \mathcal{B})$, the preimage $\mathfrak F \in \mathrm{REx}(\mathrm{mod}_\mathcal{A}, \mathcal{B})$ (\cref{eq:fp_univ_prop}) is exact. We take any short exact sequence $0 \to K \to G \to H \to 0$ in $\mathrm{mod}_\mathcal{A}$ and construct the $2$-presentation (\cref{ex:fp_proj_res}) by horseshoe lemma:
        \begin{equation}\label{eq:fp_exact_diag}
            \xymatrix{
0 \ar@{->}[r] & (-, X^{-2}) \ar@{->}[d] \ar@{->}[r] & (-, Y^{-2}) \ar@{->}[r] \ar@{->}[d] & (-, Z^{-2}) \ar@{->}[r] \ar@{->}[d] & 0 \\
0 \ar@{->}[r] & (-, X^{-1}) \ar@{->}[d] \ar@{->}[r] & (-, Y^{-1}) \ar@{->}[r] \ar@{->}[d] & (-, Z^{-1}) \ar@{->}[r] \ar@{->}[d] & 0 \\
0 \ar@{->}[r] & (-, X^0) \ar@{->>}[d] \ar@{->}[r] & (-, Y^0) \ar@{->>}[d] \ar@{->}[r] & (-, Z^0) \ar@{->>}[d] \ar@{->}[r] & 0 \\
0 \ar@{->}[r] & K \ar@{->}[r]^{i} & G \ar@{->}[r]^{p} & H \ar@{->}[r] & 0
            }.
        \end{equation}
        Applying $\mathfrak F$ to \cref{eq:fp_exact_diag}, we obtain the commutative diagram (note that $\mathfrak F$ is right exact):
        \begin{equation}
            \xymatrix{
0 \ar@{->}[r] & F(X^{-2}) \ar@{->}[d] \ar@{->}[r] & F(Y^{-2}) \ar@{->}[r] \ar@{->}[d] & F(Z^{-2}) \ar@{->}[r] \ar@{->}[d] & 0 \\
0 \ar@{->}[r] & F(X^{-1}) \ar@{->}[d] \ar@{->}[r] & F(Y^{-1}) \ar@{->}[r] \ar@{->}[d] & F(Z^{-1}) \ar@{->}[r] \ar@{->}[d] & 0 \\
0 \ar@{->}[r] & F(X^{0}) \ar@{->>}[d] \ar@{->}[r] & F(Y^{0}) \ar@{->>}[d] \ar@{->}[r] & F(Z^{0}) \ar@{->>}[d] \ar@{->}[r] & 0 \\
\ker \mathfrak F(i) \ \ar@{>->}[r] & \mathfrak F(K) \ar@{->}[r]^{\mathfrak F(i)} & \mathfrak F(G) \ar@{->}[r]^{\mathfrak F(p)} & \mathfrak F(H) \ar@{->}[r] & 0
            }.
        \end{equation}
        By snack lemma between exact complexes, $\ker \mathfrak F(i) \simeq H(F(Z^{-1}) \to F(Z^{-1}) \to F(Z^0)) = 0$.
    \end{proof}
\end{theorem}

\begin{corollary}
    Let $\mathcal{A}$ be any category.
    \begin{enumerate}
        \item For any category $\mathcal{B}$ with cokernels, the functor $F: \mathcal{A} \to \mathcal{B}$ extends uniquely (under equivalences) to a right exact functor $\mathrm{mod}_\mathcal{A} \to \mathcal{B}$ (\cref{thm:fp_univ_prop}).
        \item If $\mathcal{A}$ has weak kernels, then for any Abelian category $\mathcal{B}$, the functor $F: \mathcal{A} \to \mathcal{B}$ extends uniquely (under equivalences) to an exact functor $\mathrm{mod}_\mathcal{A} \to \mathcal{B}$ (\cref{thm:fp_exact}).
    \end{enumerate}
\end{corollary}

\begin{corollary}\label{cor:fp_left_adj}
    When $\mathcal{A}$ has weak kernels, the left adjoint of the $\mathbb Y : \mathcal{A} \to \mathrm{mod}_\mathcal{A}$ (\cref{thm:fp_univ_prop}) is exact. 
    \begin{proof}
        The weak kernel-preserving functor $\mathrm{id}_\mathcal{A} : \mathcal{A} \to \mathcal{A}$ extends to this exact functor $\mathrm{mod}_\mathcal{A} \to \mathcal{A}$. 
    \end{proof}
\end{corollary}

\begin{example}
    Let $\mathcal{T}$ be pre-triangulated. Any homological functor $H: \mathcal{T} \to \mathbf{Ab}$ (i.e., sending distinguished triangles to long exact sequences) extends to an exact functor $\mathrm{mod}_\mathcal{T} \to \mathbf{Ab}$, which is uniquely under equivalences. Note that $\mathcal{T} \to \mathrm{mod}_\mathcal{T}$ and $\mathcal{T} \to (\mathrm{mod}_{\mathcal{T}^{\mathrm{op}}})^{\mathrm{op}}$ factors through one another, the Abelian category $\mathrm{mod}_\mathcal{T}$ is Frobenius exact. The projectives are summands of representable functors.
\end{example}

\section{Auslander's Defects}

\subsection{Defects}

\begin{notation}
    Let $\mathcal{A}$ be an Abelian category, $\mathcal{E}$ consists of all short exact sequences in $\mathcal{A}$.
\end{notation}

\begin{definition}[Defect]
    Let $\delta \in \mathrm{Ext}^1(Z, X)$ be any extension, realised by a short exact sequence $X \xrightarrow f Y \xrightarrow g Z$. We set
    \begin{equation}
        0 \to (-, X) \xrightarrow{(-, f)} (-, Y) \xrightarrow{(-, g)} (-, Z) \to \delta^\ast \to 0,
    \end{equation}
    and 
    \begin{equation}
        0 \to (Z, -) \xrightarrow{(g, -)} (Y, -) \xrightarrow{(f, -)} (X, -) \to \delta_\ast \to 0.
    \end{equation}
\end{definition}

\begin{lemma}\label{lem:defect_homotopy}
    $\delta ^\ast \simeq \varepsilon ^\ast$ if and only if the corresponding shrot exact sequences are homotopy equivalent.
    \begin{proof}
        When $\delta^\ast \simeq \varepsilon^\ast$, there are induced commutative diagram between $4$-term exact sequences:
        \begin{equation}
            \xymatrix{
(-, X) \ \ar@{>->}[r] \ar@{-->}[d] & (-, Y) \ar@{->}[r] \ar@{-->}[d] & (-, Z) \ar@{->>}[r] \ar@{-->}[d] & \delta^\ast \ar@{->}[d]^{\simeq} \\
(-, A) \ \ar@{>->}[r] & (-, B) \ar@{->}[r] & (-, C) \ar@{->>}[r] & \varepsilon ^\ast
            }.
        \end{equation}
        There dashed arrows are induced by projective objects. Once we truncate the last columns, we obtain a quasi-equivalence between projective complexes, which is also a homotopy equivalence. By Yoneda lemma, we obtain a homotopy equivalence between short exact sequences in $\mathcal{E}$. The converse is clear.
    \end{proof}
\end{lemma}

\begin{notation}
    Let $\mathrm{def}_\mathcal{A}$ denotes the full subcategory of $\mathrm{mod}_\mathcal{A}$ consisting of all defects $\delta^\ast$ for $\delta \in \mathcal{E}$.
\end{notation}

\begin{theorem}
    $(-)^\ast : \mathcal{E} \to \mathrm{def}_\mathcal{A},\quad \delta \mapsto \delta^\ast$ induces an equivalence between the homotopy category of $\mathcal{E}$ and $\mathrm{def}_\mathcal{A}$.
    \begin{proof}
        By \cref{lem:defect_homotopy}, $\overline {(-)^\ast} : (\mathcal{E}/ \sim) \to \mathrm{def}(\mathcal{A})$ is a well-defined functor. $\overline {(-)^\ast}$ is also full and essentially surjective as $(-)^\ast$ is. It remains to show that $\overline {(-)^\ast}$ is faithful. Let $\delta, \varepsilon \in \mathcal{E}$ be any two extensions connected by a morphism $(\alpha, \beta, \gamma)$, suppose that $\overline {(\alpha, \beta, \gamma)^\ast } = 0$. We obtain $s^{-1}$ such that $v \circ s^{-1} = \gamma$ by cokernel $w$. Note that $\beta - s^{-1} \circ l$ is annihilated by $v$, it factors through $u$ by some $s^{-2}$. Finally, we see $s^{-2} \circ k = \alpha$ by composing a monomorphism $u$.
        \begin{equation}
            \xymatrix{
(-, X) \ \ar@{>->}[r]^{k} \ar@{-->}[d]^{\alpha} & (-, Y) \ar@{->}[r]^{l} \ar@{-->}[d]^{\beta} \ar@{->}[ld]^{s^{-2}} & (-, Z) \ar@{->>}[r]^{m} \ar@{-->}[d]^{\gamma} \ar@{->}[ld]^{s^{-1}} & \delta^\ast \ar@{->}[d]^{0} \\
(-, A) \ \ar@{>->}[r]^{u} & (-, B) \ar@{->}[r]^{v} & (-, C) \ar@{->>}[r]^{w} & \varepsilon ^\ast
            }.
        \end{equation}
    \end{proof}
\end{theorem}

\begin{corollary}\label{cor:defect_mod}
    A similar argument shows that $\mathrm{mod}_\mathcal{A}$ is the homotopy category of left exact sequences. Hence,
    \begin{equation}
        \xymatrix{
\textbf{Left exact sequences} \ar@{->>}[rr] &  & \frac{\textbf{Left exact sequences}}{\textbf{Homotopy}} \ar@{->}[rr]^{\simeq} &  & \mathrm{mod}_{\mathcal A} \\
\textbf{Exact sequences} \ar@{->>}[rr] \ar@{>->}[u] &  & \frac{\textbf{Exact sequences}}{\textbf{Homotopy}} \ar@{>->}[u] \ar@{->}[rr]^{\simeq} &  & \mathrm{def}_{\mathcal A} \ar@{>->}[u]
        }.
    \end{equation}
\end{corollary}

\begin{lemma}[Heller's exact category]\label{lem:fp_exact_str}
    $\mathcal{E}$ admits an exact structure where the conflations are  $3 \times 3$ diagrams of short exact sequences. 
    \begin{proof}
        The kernel-cokernel pairs are clear. We verify EX0, EX1, and EX2 and omit the verification of dual axioms.
        \begin{enumerate}
            \item The axiom EX0 says isomorphisms are both inflations. This is clear.
            \item The axiom EX1 says that the composition of two inflations is still a inflation. We take two composable inflations $(f_1, f_2, f_3)$ and $(g_1, g_2, g_3)$, and consider the following diagram (Noether's isomorphism):
                \begin{equation}
                \xymatrix{
                A_i \ar@{=}[d] \ar@{>->}[r]^{f_i} & B_i \ar@{->>}[r]^{u_i} \ar@{>->}[d]^{g_i} & C_i \ar@{>->}[d]^{v_i} &  &  \\
                A_i \ar@{>->}[r]^{g_i \circ f_i} & D_i \ar@{->>}[r]^{p_i} \ar@{->>}[d]^{q_i \circ p_i} & E_i \ar@{->>}[d]^{q_i} &  & i \in \{1,2,3\} \\
                & F_i \ar@{=}[r] & F_i &  & 
                }
                \end{equation}
                We explain the construction of $\text{Diagram}_1 \rightarrowtail \text{Diagram}_2\twoheadrightarrow \text{Diagram}_3$. We construct all arrows connecting to $A_\bullet$, $B_\bullet$ and $D_\bullet$, all associated squares commute by assumption. We introduce $C_1 \rightarrowtail C_2 \twoheadrightarrow C_3$, $E_1 \rightarrowtail E_2 \twoheadrightarrow E_3$ and $F_1 \rightarrowtail F_2 \twoheadrightarrow F_3$ by cokernels. These three mapping chains are conflations by $3 \times 3$ lemma. The rest of the morphisms are uniquely determined by universal properties. The commutativity are verified by composing with deflation. For instance, to see $E_1 \rightarrowtail E_2 \twoheadrightarrow F_2$ coincides with $E_1 \twoheadrightarrow F_1 \rightarrowtail F_2$, we compose $D_1 \twoheadrightarrow E_1$ forward and see the commutativity by diagram chasing.
                \item The axiom EX2 says that the pushout of an inflation along any morphism exists and is still an inflation. This axiom is easily verified in functor category.
        \end{enumerate}
    \end{proof}
\end{lemma}

\begin{example}
    Even if $\mathcal{A}$ is Abelian, $\mathcal{E}$ is not necessarily Abelian. For instance, the morphism
    \begin{equation}
        \xymatrix{
0 \ \ar@{>->}[r] \ar@{->}[d] & A \ar@{->>}[r]^{\mathrm{id}_A} \ar@{->}[d]^{\mathrm{id}_A} & A \ar@{->}[d] \\
A \ \ar@{>->}[r]^{\mathrm{id}_A} & A \ar@{->>}[r] & 0
}
    \end{equation}
    is both monic and epic, but not an isomorphism.
\end{example}

\subsection{Auslander's Defect Formula}

\begin{notation}
    We fix $\mathcal{A} := \mathrm{Mod}_R$ as the category of right modules over a unital (not necessary commutative) ring $R$. Let $E$ be any injective object (summands of $\prod\limits_{i \in I} ({}_RR, \mathbb Q / \mathbb R)_{\mathbf{Ab}}$) in $\mathrm{Mod}_R$. We introduce two contravariant functors:
    \begin{enumerate}
        \item $(-)^\star := (-, R) : \mathrm{Mod}_R \to {}_R\mathrm{Mod},\quad M \mapsto (M, R)$, and 
        \item $D(-) := (-, E) : \mathrm{Mod}_R \to {}_R\mathrm{Mod},\quad N \mapsto (N, E)$.
    \end{enumerate}
\end{notation}

\begin{lemma}
    $(-)^\star$ induce a duality restricted to finitely generated projective modules.
    \begin{proof}
        $(-)^\star$ induces an equivalence between finitely generated free modules, hence between finitely generated projective modules by taking summands.
    \end{proof}
\end{lemma}

We ask to what extent $(-)^\star$ induces a ``duality'' between $\mathrm{mod}_R$ and $\mathrm{mod}_{R^{\mathrm{op}}}$.

\begin{definition}[Auslander-Bridger transpose]
    We assume $R$ to be semi-perfect, so that any finitely generated module admits a minimal projective presentations. Let $M \in \mathrm{mod}_R$ be any finitely presented module. We take a minimal projective presentation $P^{-1} \xrightarrow f P^0 \to M \to 0$ and define the Auslander-Bridger transpose of $M$ to be $\operatorname{Tr} M := \operatorname{cok} (f^\star)$. Hence, there is a $4$-term exact sequence
    \begin{equation}
        0 \to M^\star \to (P^0)^\star \xrightarrow{f^\star} (P^{-1})^\star \to \operatorname{Tr} M \to 0.
    \end{equation}
\end{definition}

\begin{remark}
    We recollect some properties of $\operatorname{Tr}$.
    \begin{enumerate}
        \item Suppose $M \in \mathrm{mod}_R$ has no projective summands, then so is $\operatorname{Tr} M$.
        \item Suppose $M \in \mathrm{mod}_R$ has no projective summands. If and only if $f$ is a minimal projective presentation of $M$, $f^\star$ is a minimal projective presentation of $\operatorname{Tr} M$.
        \item Suppose $M \in \mathrm{mod}_R$ has no projective summands, then $\operatorname{Tr} \operatorname{Tr} M \simeq M$.
        \item Let $f$ and $g$ be any two projective presentations of $M$, then $\operatorname{cok} (f^\star) \oplus P \simeq \operatorname{cok} (g^\star) \oplus Q$ for some projective modules $P, Q$. Moreover, $\mathrm{Tr}$ introduces a duality between stable categories $\underline{\mathrm{mod}}_R$ and $\underline{_R\mathrm{mod}}$.
    \end{enumerate}
    \begin{proof}
        The proof concerns with radical theory. See \S 32 of \cite{anderson2012rings}.
    \end{proof}
\end{remark}

\begin{example}\label{ex:defect_tr}
    For any $M \in \mathrm{mod}_R$ with minimal presentation $M = \operatorname{cok} f$ and $A \in \mathrm{Mod}_R$, there is a $4$-term exact sequnce 
    \begin{equation}
        \xymatrix{
 &  & A \otimes (P^{0})^\star \ar@{->}[r]^{A \otimes f^\star} \ar@{->}[d]^{\simeq} & A \otimes (P^{-1})^\star \ar@{->}[r]^{A \otimes p} \ar@{->}[d]^{\simeq} & A \otimes \operatorname {Tr}M \ar@{->}[r] & 0 \\
0 \ar@{->}[r] & (M, A) \ar@{->}[r]^{{(p,A)}} & (P^0, A) \ar@{->}[r]^{{(f, A)}} & (P^{-1},A) &  & 
        }.
    \end{equation}
\end{example}

\begin{theorem}[Auslander's defect formula]
    Let $A \xrightarrow f B \xrightarrow g C$ be any short exact sequence in $\mathrm{Mod}_R$ which corresponds to an extension $\delta \in \mathrm{Ext}^1(C,A)$. Then there is a natural isomorphism $D \delta^\ast \simeq \delta_\ast (D\operatorname{Tr} X)$ for any $X \in \mathrm{mod}_R$.
    \begin{proof}
        We apply the $4$-term exact sequence in \cref{ex:defect_tr} to the short exact sequence, and obtain the following commutative diagram with exact rows and columns by snake lemma:
        \begin{equation}
            \xymatrix{
 &  &  & \delta^\ast (X) \ar@{>->}[d] \\
(X, A) \ar@{>->}[r] \ar@{>->}[d] & (P^0, A) \ar@{->}[r] \ar@{>->}[d] & (P^{-1},A) \ar@{->>}[r] \ar@{>->}[d] & A \otimes \operatorname {Tr}X \ar@{->}[d] \\
(X,B) \ar@{>->}[r] \ar@{->}[d] & (P^0,B) \ar@{->}[r] \ar@{->>}[d] & (P^{-1},B) \ar@{->>}[r] \ar@{->>}[d] & B \otimes \operatorname {Tr}X \ar@{->>}[d] \\
(X,C) \ar@{>->}[r] \ar@{->>}[d] & (P^0,C) \ar@{->}[r] & (P^{-1},C) \ar@{->>}[r] & C \otimes \operatorname {Tr}X \\
\delta^\ast (X) &  &  & 
}
        \end{equation}
        Hence, 
        \begin{equation}
            D(\delta^\ast X) \simeq D(\ker (f \otimes \operatorname{Tr}X)) \simeq \operatorname{cok} (D(f \otimes \operatorname{Tr}X)) \simeq \operatorname{cok} (f, \operatorname{Tr}X) \simeq \delta_\ast (D\operatorname{Tr}X).
        \end{equation}
    \end{proof} 
\end{theorem}

\begin{corollary}
    Note that $\delta_\ast (D R) = 0$ as $DR$ is injective. We see $D(\delta^\ast X) \simeq \delta_\ast (DX^\triangle)$, where $X^\triangle \simeq \operatorname{cok} p^\star$ for any projective presentation $P^{-1} \xrightarrow p P^0 \to X \to 0$.
\end{corollary}

\begin{notation}
    We write $F(\operatorname{Tr}X)$ when $X \in \mathrm{mod}_R$ and all $F(X^\triangle )$ takes isomorphic values, even though there is no minimal projective presentation of $X$. For instance, $D(\delta^\ast X) \simeq \delta_\ast (D\operatorname{Tr}X)$ over any ring.
\end{notation}

\begin{theorem}[Auslander-Reiten formula]
    For any $X \in \mathrm{mod}_R$, there are natural isomorphisms
    \begin{equation}\label{eq:AR_formula}
        D\underline {(X, -)} \simeq \operatorname{Ext}^1(-, D\operatorname{Tr} X) : \mathrm{Mod}_R^{\mathrm{op}} \to \mathbf{Ab},
    \end{equation}
    and 
    \begin{equation}\label{eq:AR_formula_2}
        \overline {(-, D\operatorname{Tr}X)} \simeq D\mathrm{Ext}^1(X, -) : \mathrm{Mod}_R^{\mathrm{op}} \to \mathbf{Ab}.
    \end{equation}
    \begin{proof}
        We show \cref{eq:AR_formula}. For any $C \in \mathrm{Mod}_R$ and short exact sequence $\delta : 0 \to K \to P \to C \to 0$ with $P$ projective, we see
        \begin{equation}
            \delta^\ast (X) = \operatorname{cok} ((X, P) \to (X, C)) \simeq \underline{(X, C)}.
        \end{equation}
        Hence, $D\underline{(X, C)} \simeq D(\delta^\ast X) \simeq \delta_\ast (D\operatorname{Tr}X) \simeq \operatorname{Ext}^1(C, D\operatorname{Tr}X)$. One can show \cref{eq:AR_formula_2} by injective pre-envelopes.
    \end{proof}
\end{theorem}

\begin{remark}
    In context of AR theory, $\tau := D\operatorname{Tr}$ is known as the AR translation.
\end{remark}

\subsection{Locally Effaceable Functors}

\begin{definition}[Locally Effaceable functor]\label{def:loc_efface}
    Let $\mathcal{A}$ be any additive category. Say $F : \mathcal{A}^{\mathrm{op}} \to \mathbf{Ab}$ is locally effaceable if for any object $X \in \mathcal{A}$ and any element $x \in FX$, there is an epimorphism $p : Y \to X$ such that $F(p)(x) = 0$.
\end{definition}

\begin{notation}
    We write $\mathrm{Eff}_\mathcal{A}$ for the full subcategory of $\mathrm{Mod}_\mathcal{A}$ consisting of all locally effaceable functors.
\end{notation}

\begin{remark}
    Some one call \cref{def:loc_efface} ``coeffacable''.
\end{remark}

\begin{example}
    How one understands \cref{def:loc_efface}? Note that an epimorphism $p : Y \to X$ (especially in the concrete categories) creates phantom information. For instance, the epimorphism $p : \mathbb R \twoheadrightarrow \mathbb R / \mathbb Z$ creates loops. Under a loop-detecting functor $H_1$, the phantom information $[w] \in H_1(\mathbb R / \mathbb Z)$ is killed by $H_1(p)([w]) = 0$.
\end{example}

\begin{lemma}\label{lem:efface_serre}
    $\mathrm{Eff}_\mathcal{A}$ is a Serre subcategory of $\mathrm{Mod}_\mathcal{A}$.
    \begin{proof}
        We verify that $\mathrm{Eff}_\mathcal{A}$ is closed under subobjects, quotients and extensions. Equivalently, for any mapping sequence $F \xrightarrow{\iota} G \xrightarrow{\pi} H$ exact at $G$, if $F, H \in \mathrm{Eff}_\mathcal{A}$, then so is $G$.
        
        For any $X \in \mathcal{A}$ and $m \in GX$, we have $\pi_X(m) \in HX$. By exactness, there is an epimorphism $p : Y \to X$ such that $H(p)(\pi_X(m)) = 0$. Hence, $\pi_Y(G(p)(m)) = H(p)(\pi_X(m)) = 0$. There is $\widetilde m \in FY$ such that $\iota_Y(\widetilde m) = G(p)(m)$. Since $F \in \mathrm{Eff}_\mathcal{A}$, there is an epimorphism $q : Z \to Y$ such that $F(q)(\widetilde m) = 0$. We see $(p \circ q) : Z \to X$ is an epimorphism such that $G(p \circ q)(m) = G(q)(G(p)(m)) = G(q)(\iota_Y(\widetilde m)) = \iota_Z(F(q)(\widetilde m)) = 0$. We see $G \in \mathrm{Eff}_\mathcal{A}$.
        % \begin{enumerate}
            % \item (Closed under subobjects). Let $G \in \mathrm{Eff}_\mathcal{A}$ and $\iota : F \rightarrowtail G$ be any subfunctor. For any $X \in \mathcal{A}$ and $m \in FX$, there is an epimorphism $p : Y \to X$ such that $G(p)(\iota _X (m)) = 0$. Hence, $\iota_Y (F(p)(m)) = 0$. Since $\iota_Y$ is monic, $F(p)(m) = 0$. We see $F \in \mathrm{Eff}_\mathcal{A}$.
            % \item (Closed under quotients). Let $G \in \mathrm{Eff}_\mathcal{A}$ and $\pi : G \twoheadrightarrow H$ be any quotient functor. For any $X \in \mathcal{A}$ and $m \in HX$, there is $\widetilde m \in GX$ such that $\pi_X(\widetilde m) = m$. There is an epimorphism $p : Y \to X$ such that $G(p)(\widetilde m) = 0$. Hence, $H(p)(m) = H(p)(\pi_X(\widetilde m)) = \pi_Y(G(p)(\widetilde m)) = 0$. We see $H \in \mathrm{Eff}_\mathcal{A}$.
            % \item (Closed under extensions). Let $0 \to F \xrightarrow{\iota} G \xrightarrow{\pi} H \to 0$ be any short exact sequence with $F, H \in \mathrm{Eff}_\mathcal{A}$. For any $X \in \mathcal{A}$ and $m \in GX$, we have $\pi_X(m) \in HX$. By exactness, there is an epimorphism $p : Y \to X$ such that $H(p)(\pi_X(m)) = 0$. Hence, $\pi_Y(G(p)(m)) = H(p)(\pi_X(m)) = 0$. There is $\widetilde m \in FY$ such that $\iota_Y(\widetilde m) = G(p)(m)$. Since $F \in \mathrm{Eff}_\mathcal{A}$, there is an epimorphism $q : Z \to Y$ such that $F(q)(\widetilde m) = 0$. We see $(p \circ q) : Z \to X$ is an epimorphism such that $G(p \circ q)(m) = G(q)(G(p)(m)) = G(q)(\iota_Y(\widetilde m)) = \iota_Z(F(q)(\widetilde m)) = 0$. We see $G \in \mathrm{Eff}_\mathcal{A}$.
        % \end{enumerate}
    \end{proof}
\end{lemma}

We analysis a special kind of locally effaceable functors.

\begin{proposition}\label{prop:epi_efface}
    If $F:= \operatorname{cok} (-, p) \in \mathrm{Eff}_\mathcal{A}$, then $p$ is epic.
    \begin{proof}
        We take the universal element $u := \theta_Y(\mathrm{id}_Y) \in F(p) \simeq [(-, Y), F] \ni \theta$ in the presentation
        \begin{equation}
            (-, X) \xrightarrow{(-, p)} (-, Y) \xrightarrow{\theta} F \to 0.
        \end{equation}
        By assumption, there is an epimorphism $q : T \twoheadrightarrow Y$ such that $F(q)(u) = 0$. We see $0 = F(q)(u) = \theta_T(q)$, hence $q \in \operatorname{im} ((T, p))$. There is $r \in (T, X)$ such that $p \circ r = q$. Since $q$ is epic, so is $p$.
    \end{proof}
\end{proposition}

\begin{definition}\label{def:epi_complete}
    We define a property for category that, any cospan of an epimorphism and any morphism can be completed to a commutative square with an epimorphism as the other leg. More precisely, any diagram on the left hand side completes to a commutative diagram on the right hand side ($\twoheadrightarrow$ for epimorphisms)
    \begin{equation}
        \xymatrix{
 & \cdot \ar@{->>}[d] &  & \cdot \ar@{-->}[r] \ar@{-->>}[d] & \cdot \ar@{->>}[d] \\
\cdot \ar@{->}[r] & \cdot &  & \cdot \ar@{->}[r] & \cdot
        }.
    \end{equation}
\end{definition}

\begin{remark}
    The category with enough projectives, exact categories, triangulated categories has property \cref{def:epi_complete}.
\end{remark}

\begin{proposition}\label{prop:epi_complete}
    Suppose $\mathcal{A}$ has the property \cref{def:epi_complete}. If $p \in \mathcal{A}$ is an epimorphism, then $F:= \operatorname{cok} (-, p) \in \mathrm{Eff}_\mathcal{A}$.
    \begin{proof}
        We take any $M \in \mathcal{A}$ and $m \in FM$. There is $f \in (M, Y)$ such that $\theta_M(f) = m$. We take the following commutative diagram:
        \begin{equation}
\xymatrix{
 &  & f\circ q \ar@{|-->}[rr] &  & \theta_L (f \circ q) \ar@{=}[rd] &  \\
L \ar@{-->}[r]^{g} \ar@{-->>}[d]^{q} & X \ar@{->>}[d]^{p} &  & (L,Y) \ar@{->>}[r]^{\theta_L} & FL & (Fq)(m) \\
M \ar@{->}[r]^{f} & Y &  & (M,Y) \ar@{->>}[r]^{\theta_M} \ar@{->}[u]_{{(M,q)}} & FM \ar@{->}[u]_{Fq} &  \\
 &  & f \ar@{|->}[rrr] \ar@{|->}[uuu] &  &  & m \ar@{|->}[uu]
}.
        \end{equation}
        Note that $\theta_L (f \circ q) = \theta_L (p \circ g) \subseteq \operatorname{im} (\theta _L \circ (L, p)) = 0$. Hence, $(Fq)(m) = 0$.
    \end{proof}
\end{proposition}

\begin{notation}
    We write $\mathrm{eff}_\mathcal{A} := \mathrm{Eff}_\mathcal{A} \cap \mathrm{mod}_\mathcal{A}$ for the full subcategory of $\mathrm{mod}_\mathcal{A}$ consisting of all locally effaceable finitely presented functors.
\end{notation}

\begin{theorem}\label{thm:epi_efface}
    Under assumption of \cref{def:epi_complete}, $\mathrm{eff}_\mathcal{A}$ consists of $\operatorname{cok} (-, p)$ for some epimorphism $p$.
    \begin{proof}
        By \cref{prop:epi_efface} and \cref{prop:epi_complete}.
    \end{proof}
\end{theorem}

\begin{remark}
    $\mathrm{def}_\mathcal{A} \subseteq \mathrm{eff}_\mathcal{A}$ when $\mathcal{A}$ is exact. The converse is not true in general, at least for non-idempotent complete ones.
\end{remark}

\begin{corollary}
    $\mathrm{eff}_\mathcal{A}$ is closed under cokernels and extensions (pointwise computed).
    \begin{proof}
        The cokernels and extensions in both $\mathrm{mod}_\mathcal{A}$ and $\mathrm{Eff}_\mathcal{A}$ are pointwise computed.
    \end{proof}
\end{corollary}

\begin{corollary}\label{cor:efface_abelian}
    When $\mathcal{A}$ has weak kernels, $\mathrm{eff}_\mathcal{A}$ is Abelian (pointwise computed). $\mathrm{eff}_\mathcal{A}$ is also a Serre subcategory of $\mathrm{mod}_\mathcal{A}$.
        \begin{proof}
        The kernels, cokernels and extensions in both $\mathrm{mod}_\mathcal{A}$ and $\mathrm{Eff}_\mathcal{A}$ are pointwise computed, thus $\operatorname{coim} \simeq \operatorname{im}$ holds in $\mathrm{eff}_\mathcal{A}$. It yields that $\mathrm{eff}_\mathcal{A}$ is Abelian.

        We show $\mathrm{eff}_\mathcal{A}$ is a Serre subcategory of $\mathrm{mod}_\mathcal{A}$. For any functor $L \in \mathrm{mod}_\mathcal{A}$ obtained by taking either taking subobjects, quotients or extensions in $\mathrm{mod}_\mathcal{A}$ of functors in $\mathrm{eff}_\mathcal{A}$, we see $F$ belongs to both $\mathrm{mod}_\mathcal{A}$ and $\mathrm{Eff}_\mathcal{A}$, hence $L \in \mathrm{eff}_\mathcal{A}$.
            \begin{equation}\label{eq:effaceable_serre}
\xymatrix{
\ \mathrm{eff}_\mathcal A\ : \ar@{=}[r] \ar@{>->}[d]^{\text{Serre}} & \mathrm{Eff}_\mathcal A \cap \mathrm{mod}_\mathcal A \ar@{>->}[d] \ar@{>->}[r] & \mathrm{Eff}_\mathcal A \ar@{>->}[d]^{\text{Serre}} \\
\mathrm{mod}_\mathcal A \ar@{=}[r] & \mathrm{mod}_\mathcal A \ar@{>->}[r] & \mathrm{Mod}_\mathcal A
}.
    \end{equation}
    \end{proof}
\end{corollary}

% \begin{proposition}
%     $0$ is the only projective object in $\mathrm{eff}_\mathcal{A}$.
%     \begin{proof}
%         When $F \simeq \operatorname{cok} (-, f)$ is projective, then $f$ is epic by \cref{prop:epi_efface}. We obtain
%         \begin{equation}
% \xymatrix{
% (-,X) \ar@{->>}[rd]^{q} \ar@{->}[rr]^{{(-,f)}} &  & (-,Y) \ar@{->>}[r] \ar@/^1pc/@{-->>}[ld]^{p} & F \\
%  & K \ar@{>->}[ru]^{i} \ar@/^1pc/@{>->}[lu]^{j} &  & 
% }.
%         \end{equation}
%         Here $i$ lies in split exact sequence. Since $K$ is projective, induced morphism $q$ is split epic. We take $j$ and $p$ such that $pi = \mathrm{id}_K = qj$. We write $a = iq$ and $b = jq$, then $ab$ and $ba$ are idempotent.
%     \end{proof}
% \end{proposition}

\subsection{Admissible Locally Effaceable Functors}

For exact categories, non-admissible epimorphisms are usually viewed as a usual map. The definition of locally effaceable functors includes unusable ones. We introduce admissible locally effaceable functors to avoid this issue. The theory is established over extriangulated categories (see \cite{nakaokaExtriangulatedCategoriesHovey2019} for an introduction).

\begin{notation}
    An extriangulated category is a triplet $(\mathcal{A}, \mathbb E, \mathfrak s)$, where $\mathcal{A}$ is an additive category, $\mathbb E : \mathcal{A}^{\mathrm{op}} \times \mathcal{A} \to \mathbf{Ab}$ is an additive bifunctor, and $\mathfrak s$ is a realization of $\mathbb E$ satisfying certain axioms. For any $\delta \in \mathbb E(Z, X)$, aka an $\mathbb E$-extension, we write a realisation of $\delta$ as $X \overset f \rightarrowtail Y \overset g\twoheadrightarrow Z\overset \delta \dashrightarrow$, aka an $\mathbb E$-conflation or an $\mathbb E$-triangle. We call $f$ an $\mathbb R$-inflation and $g$ an $\mathbb E$-deflation.
\end{notation}

\begin{example}
    For any $\mathbb E$-triangle $X \overset f \rightarrowtail Y \overset g\twoheadrightarrow Z\overset \delta \dashrightarrow$, there are long exact sequences with $6$ terms, i.e.,
    \begin{equation}
        (-, X) \xrightarrow{(-, f)} (-, Y) \xrightarrow{(-, g)} (-, Z) \xrightarrow{\delta_\sharp} \mathbb E(-, X) \xrightarrow{\mathbb E(-, f)} \mathbb E(-, Y) \xrightarrow{\mathbb E(-, g)} \mathbb E(-, Z),
    \end{equation}
    and 
    \begin{equation}\label{eq:long_exact_6}
        (Z, -) \xrightarrow{(g, -)} (Y, -) \xrightarrow{(f, -)} (X, -) \xrightarrow{\delta^\sharp} \mathbb E(Z, -) \xrightarrow{\mathbb E(g, -)} \mathbb E(Y, -) \xrightarrow{\mathbb E(f, -)} \mathbb E(X, -).
    \end{equation}
    Hence, $f$ is a weak kernel of $g$ and $g$ is a weak cokernel of $f$.
\end{example}

We show that WIC condition (\textbf{Condition 5.8}, \cite{nakaokaExtriangulatedCategoriesHovey2019}) is the same as weakly idempotent completeness. 

\begin{proposition}
    An extriangulated category is a weakly idempotent complete additive category, if and only if, for any inflation of the form $f \circ i$, $i$ is also an inflation.
    \begin{proof}
        We assume $\mathcal{A}$ is idempotent complete. By \textbf{Corollary 3.16} in \cite{nakaokaExtriangulatedCategoriesHovey2019}, $\binom {f \circ i}{i}$ is an inflation. Hence, $\binom i 0$ is an inflation by composing an isomorphism. By \cref{eq:long_exact_6}, we obtain $\binom{a}{b}$ as in the following commutative diagram:
        \begin{equation}
\xymatrix{
X \ \ar@{>->}[r]^{\binom i0} \ar@/_1pc/@{->}[rd]_{0} & Y \oplus Z \ar@{->>}[r]^{{(s,t)}} \ar@{->}[d]^{{\binom{s,0}{0,1}}} & W \ar@/^1pc/@{.>}[ld]^{\binom ab} \\
 & W \oplus Z & 
}.
        \end{equation}
        Hence, $t$ is split monic. We identify $t = \binom 01 : Q \oplus Z$ by assumption. By \textbf{Proposition 3.17} in \cite{nakaokaExtriangulatedCategoriesHovey2019}, we obtain a commutative diagram of $\mathbb E$-triangles:
        \begin{equation}
            \xymatrix{
 & Z \ar@{=}[r] \ar@{>->}[d]^{\binom 01} & Z \ar@{>->}[d]^{\binom 01} \\
X \ar@{>->}[r]^{\binom i0} \ar@{=}[d] & Y \oplus Z \ar@{->>}[r]^{{\binom{s_1,0}{s_2,1}}} \ar@{->>}[d]^{{(1,0)}} & Q \oplus Z \ar@{->>}[d]^{{(1,0)}} \\
X \ar@{>-->}[r]^{i} & Y \ar@{-->}[r]^{s_1} & Q
            }.
        \end{equation}
        We see $i$ is an inflation.
        
        Conversely, for any split monomorphism $i : X \to Y$, there is $p : Y \to X$ such that $pi = \mathrm{id}_X$. Since $\mathrm{id}_X$ is an inflation, so is $i$ by assumption. $\operatorname{cok} i$ is clearly found in the corresponding $\mathbb E$-triangle. We see $\mathcal{A}$ is idempotent complete.
    \end{proof}
\end{proposition}

\begin{definition}[Admissible locally effaceable functor]
    Say $F \in \mathrm{Mod}_\mathcal{A}$ is admissible locally effaceable if for any $X \in \mathcal{A}$ and any $x \in FX$, there is an $\mathbb E$-deflation $p : Y \twoheadrightarrow X$ such that $F(p)(x) = 0$.
\end{definition}

\begin{notation}
    We write $\mathrm{Eff}^a_\mathcal{A}$ for the full subcategory of $\mathrm{Mod}_\mathcal{A}$ consisting of all admissible locally effaceable functors.
\end{notation}

\begin{remark}
    $\mathrm{Eff}^a_\mathcal{A} \subsetneqq \mathrm{Eff}_\mathcal{A}$, since an $\mathbb E$-deflation is not an epimorphism in general.
\end{remark}

\begin{corollary}
    $\mathrm{Eff}^a_\mathcal{A}$ is also a Serre subcategory $\mathrm{Mod}_\mathcal{A}$. In fact, the proof of \cref{lem:efface_serre} is valid by replacing epimorphisms with any collection of morphisms closed under compositions.
\end{corollary}

\begin{corollary}\label{cor:admissible_epi}
    Assuming the weakly idempotent completeness of $\mathcal{A}$, $\mathrm{Eff}^a_\mathcal{A} \cap \mathrm{mod}_\mathcal{A} = \{\operatorname{cok}(-, p) \mid p \ \text{is an $\mathbb{E}$-deflation}\}$. The proof of \cref{prop:epi_efface} is valid when $\mathcal{A}$ is weakly idempotent completeness; the proof of \cref{prop:epi_complete} is still valid by ET3 axiom.
\end{corollary}

\begin{definition}[$\mathbb E$-defects]\label{def:defect}
    We set $\delta^\ast := \operatorname{im}(\delta_\sharp)$ and $\delta_\ast := \operatorname{im}(\delta^\sharp)$ for any $\mathbb E$-extension $\delta \in \mathbb E(Z, X)$.
\end{definition}

\begin{notation}
    We write $\mathrm{def}^a_\mathcal{A}$ for the full subcategory of $\mathrm{mod}_\mathcal{A}$ consisting of all admissible $\mathbb E$-defects.
\end{notation}

\begin{theorem}\label{thm:defect_efface}
    $\mathrm{def}^a_\mathcal{A} \simeq (\mathrm{Eff}_\mathcal{A}^a \cap \mathrm{mod}_\mathcal{A})$.
    \begin{proof}
        By \cref{cor:admissible_epi}, $\{\operatorname{cok}(-, p) \mid p \ \text{is an $\mathbb{E}$-deflation}\}$ is precisely $\mathrm{Eff}_\mathcal{A}^a \cap \mathrm{mod}_\mathcal{A}$.
    \end{proof}
\end{theorem}

\begin{notation}
    $\mathrm{eff}^a_\mathcal{A} := \mathrm{Eff}^a_\mathcal{A} \cap \mathrm{mod}_\mathcal{A}$, this notion is analogous to \cref{eq:effaceable_serre}.
\end{notation}

\begin{corollary}
    Assuming $\mathcal{A}$ has weak kernels. 
    \begin{enumerate}
        \item $\mathrm{eff}^a_\mathcal{A}$ is Abelian (pointwise computed); $\mathrm{eff}^a_\mathcal{A}$ is also a Serre subcategory of $\mathrm{mod}_\mathcal{A}$ (\cref{cor:efface_abelian}).
        \item $\mathrm{eff}^a_\mathcal{A} \simeq \mathrm{def}^a_\mathcal{A} \simeq \frac{\textbf{$\mathbb E$-conflations}}{\textbf{Homotopy}}$ (\cref{thm:defect_efface} and \cref{cor:defect_mod}). 
    \end{enumerate}
\end{corollary}

\section{Serre Subcategories}

\subsection{Abelian Localisations}

We begin with a general construction of localisations.

\begin{definition}[Localisation]\label{def:localisation}
    Let $\mathcal{C}$ be any category and $S$ be any collection of morphisms in $\mathcal{C}$. Say $Q: \mathcal{C} \to \mathcal{C}_1$ is a localisation of $\mathcal{C}$ with respect to $S$ if the following is an isomorphism (not merely an equivalence) of functor categories:
    \begin{equation}\label{eq:localisation}
        Q^\ast : \mathrm{Fun}(\mathcal{C}_1, \mathcal{D}) \xrightarrow{\simeq} \mathrm{Fun}_S(\mathcal{C}, \mathcal{D}),\quad F \mapsto F \circ Q\quad (\forall \mathcal{D}).
    \end{equation}
    Here $\mathrm{Fun}_S(\mathcal{C}, \mathcal{D}) := \{F \in \mathrm{Fun}(\mathcal{C}, \mathcal{D}) \mid F(s) \text{ is an isomorphism for all } s \in S\}$.
\end{definition}

\begin{example}[Gabriel-Zisman's approach]\label{ex:gz_localisation}
    The construction is seperated into two steps: $\mathcal{C} \xrightarrow \iota \mathcal{C}_S \xrightarrow \pi (\mathcal{C}_S/\sim )$.
    \begin{enumerate}
        \item (Construction of $\iota$). The category $\mathcal{C}_S$ consists of the same objects as $\mathcal{C}$. We take the assignment $\langle X, Y\rangle := (X, Y)_\mathcal{C} \sqcup (S \cap (Y,X))_\mathcal{C}$. We write $A \sqcup B = \{(0,a), (1,b) \mid a \in A, b \in B\}$ for a disjoint union. Let $(X,Y)_{\mathcal{C}_S}$ be freely generated by these letters, that is, the collection of finite compositions of $\langle X, Y\rangle := (X, Y)_\mathcal{C} \sqcup (S \cap (Y,X))_\mathcal{C}$. We identify $(0, \mathrm{id}_X)$ as an identical morphism in $\mathcal{C}_S$, and $(0, g \circ f)$ as $(0, g) \circ (0,f)$. Now, $\iota : \mathcal{C} \to \mathcal{C} _S$ is a functor.
        \item (Construction of $\pi$). Let $\sim$ be an equivalence relation of morphisms generated by $(1, s) \circ (0,s) \sim (0,\mathrm{id}_X)$ and $(0,s) \circ (1,s) \sim (0, \mathrm{id}_Y)$ for any $[X \xrightarrow s Y] \in S$.
    \end{enumerate}
    We take $Q = \pi \circ \iota : \mathcal{C} \to (\mathcal{C}_S/ \sim)$, and write $\pi ((n, f)) = : [n, f]$.
\end{example}

\begin{remark}\label{rem:set_theoretic_issue}
    The quotient $\pi$ is well-defined in sense of the axiom of vonNeumann-Bernays-Gödel and the axiom of choice, whereas it is impossible to determined whether $Q(f) = Q(f')$ set-theoritically. 
\end{remark}

\begin{theorem}
    The construction \cref{ex:gz_localisation} is a localisation in sense of \cref{def:localisation}.
    \begin{proof}
        Note that $F \circ Q$ maps $s$ to an isomorphism (with inverse $F([1,s])$) for any $s \in S$. We obtain a functor
        \begin{equation}
                Q^\ast : \mathrm{Fun}(\mathcal{C}_S / \sim, \mathcal{D}) \xrightarrow{\simeq} \mathrm{Fun}_S(\mathcal{C}, \mathcal{D}),\quad F \mapsto F \circ \pi \circ \iota.
        \end{equation}
        To see that $Q^\ast$ is essentially surjective, we take any $G \in \mathrm{Fun}_S(\mathcal{C}, \mathcal{D})$. We define $\widetilde G : \mathcal{C}_S \to \mathcal{D}$ as by $X \mapsto GX$, $(0,f) \mapsto Gf$, and $(1,s) \mapsto (Gs)^{-1}$. Note that $\widetilde F$ is a functor as it satisfies the rule of identities and compositions. The functor passes through the quotient $\pi$ since $\widetilde G(1,s)$ and $\widetilde G (0,s)$ are mutually inverse for any $s \in S$.

        We show $Q^\ast$ is full. Note that a natural transformation $\eta : F \to G$ in $\mathrm{Fun}_S(\mathcal{C}, \mathcal{D})$ is labelled by $\mathsf{Ob}(\mathcal{C}) = \mathsf{Ob}(\mathcal{C}_S/\sim)$. For any $s \in S$, one has 
        \begin{equation}
            \xymatrix{
\widetilde GY \ar@{->}[r]^{\eta_Y} \ar@{->}[d]^{(\widetilde Gs)^{-1}} & \widetilde FY \ar@{->}[d]^{(\widetilde Fs)^{-1}} & Y \\
\widetilde GX \ar@{->}[r]^{\eta_X} & \widetilde FX & X \ar@{->}[u]^{s}
            }.
        \end{equation}
        Hence, $\eta$ induces commutative squares for zig-zag diagrams. Now $\eta$ is a natural transformation $\widetilde F \to \widetilde G$ in $\mathrm{Fun}(\mathcal{C}_S, \mathcal{D})$. The commutative diagrams passes through the quotient $\pi$. Hence, $\eta$ is also a natural transformation of functors in $\mathrm{Fun}(\mathcal{C}_S/\sim, \mathcal{D})$, yielding the fullness of $Q^\ast$.

        We show $Q^\ast$ is faithful. Note that $Q^\ast$ never changes a collection of morphisms in $\mathcal{D}$ labelled by $\mathsf{Ob}(\mathcal{C}) = \mathsf{Ob}(\mathcal{C}_S/\sim)$.

        We finally show that $Q^\ast$ is injective on objects. Let $F, G \in \mathrm{Fun}(\mathcal{C}_S/\sim, \mathcal{D})$ such that $F \circ Q = G \circ Q$. Then $\{\mathrm{id}_X\} : F \circ Q \to G \circ Q$ is a natural isomorphism. By previous discussion on fully faithfulness, there are unique pair of inversal natural isomorphisms $\eta : F \to G$ and $\theta : G \to F$. By construction, $\eta$ and $\theta$ consists of identical morphisms. Hence $F = G$.
    \end{proof}
\end{theorem}

\begin{theorem}[Universal property of localisation]\label{thm:universal_property_localisation}
    We denote the localisation (in sense of \cref{def:localisation}) by a functor $Q : \mathcal{C} \to \mathcal{C} [S^{-1}]$. For any functor $F : \mathcal{C} \to \mathcal{D}$ that maps $s$ to isomorphisms for all $s \in S$, there is a unique functor $\widetilde F : \mathcal{C}[S^{-1}] \to \mathcal{D}$ such that $F = \widetilde F \circ Q$.
    \begin{proof}
        This is a rephrasing of \cref{def:localisation}.
    \end{proof}
\end{theorem}

By \cref{rem:set_theoretic_issue}, it is somehow impossible to determine whether two morphisms in localisation categories are equal. When the localisation admits fractional calculation, each $(X,Y)_{\mathcal{C}[S^{-1}]}$ is a (large) filtrant colimit of sets. Hence, one can determine whether two morphisms are equal within a set in the filtrant system.

\begin{definition}[Multiplicative system]\label{def:multiplicative_system}
    A collection $S$ of morphisms in $\mathcal{C}$ is a left multiplicative system if it satisfies the following axioms:
    \begin{enumerate}
        \item (LMS1). $S$ is closed under compositions and contains all identities.
        \item (LMS2). Any diagram $X' \xleftarrow s X \xrightarrow f Y$ with $s \in S$ completes to a commutative square with $t \in S$ as the other leg.
        \item (LMS3). For any pair of morphisms $f, g : X \to Y$, there is $s : Y \to Z$ in $S$ such that $sf = sg$, if and only if, there is $t : W \to X$ in $S$ such that $ft = gt$.
    \end{enumerate}
    One has right multiplicative systems by dualising the above axioms.
\end{definition}

\begin{theorem}\label{thm:localisation_fraction}
    Assume $\mathcal{C}$ admits a left multiplicative system $S$ (\cref{def:multiplicative_system}). Say two fractions $(s,f)$ and $(s',f')$ are equivalent, provided there are $t, t' \in S$ such that $ts = t's' \in S$ and $tf = t'f'$. Then it is an equivalence relation. Now the class of morphism is characterised by the filtrant colimit
    \begin{equation}
        (X,Y)_{\mathcal{C}[S^{-1}]} := \varinjlim_{(s : X \to X') \in S} (X', Y)_\mathcal{C}.
    \end{equation}
    When $\mathcal{C}$ admits a right multiplicative system, then the class of morphism is characterised by the filtrant colimit
    \begin{equation}\label{eq:right_fraction}
        (X,Y)_{\mathcal{C}[S^{-1}]} := \varinjlim_{(t : Y' \to Y) \in S} (X, Y')_\mathcal{C}.
    \end{equation}
    The resulting categories are additive, and the localisation functor $Q : \mathcal{C} \to \mathcal{C}[S^{-1}]$ is also additive.
    \begin{proof}
        The general theory is found in \cite{gabrielCalculusFractionsHomotopy1967}. 
    \end{proof}
\end{theorem}

\begin{example}[Localisation of Serre subcategory]
    Let $\mathcal{A}$ be an Abelian category and $\mathcal{C}$ be a Serre subcategory. We take $S := \{f \in \mathcal{A} \mid \operatorname{ker} f, \operatorname{cok} f \in \mathcal{C}\}$. Then $S$ is a multiplicative system. Moreover, 
    \begin{equation}\label{eq:serre_quotient_colimit}
        (X,Y)_{\mathcal{A}[S^{-1}]} = \varinjlim_{X/X', Y' \in \mathcal{C}} (X', Y/Y')_\mathcal{A}.
    \end{equation}
    The morphisms in $(X,Y)_{\mathcal{A}[S^{-1}]}$ is approached by $(X', Y/Y')_\mathcal{A}$ in the following commutative diagram:
    \begin{equation}\label{eq:serre_quotient}
\xymatrix{
X'\  \ar@{->}[d] \ar@{>->}[r]^{\in S} & X \ar@{->>}[r] \ar@{.>}[d] & X/X' & X/X' \ (\in \mathcal C) \\
Y/Y' & Y \ar@{->>}[l]^{\in S} & \ Y' \ar@{>->}[l] & Y' \ (\in \mathcal C)
}.
    \end{equation}
\end{example}

\begin{notation}
    We usually denote the Serre quotient by $\mathcal{A}/\mathcal{C}$. 
\end{notation}

\begin{lemma}\label{lem:zero_morphism_quotient}
    Let $Q : \mathcal{A} \to \mathcal{A}/\mathcal{C}$ be the Serre quotient. For any $f \in \mathsf{Mor}(\mathcal{A})$, $Qf$ is a zero morphism if and only if $f$ passes through an object in $\mathcal{C}$.
    \begin{proof}
        $Q : \mathcal{C} \to 0$ is clear. Conversely, if $Qf = 0$, then there is a zero composition
        \begin{equation}
            \xymatrix{
X'\  \ar@{>->}[r]^{i} & X \ar@{->}[r]^{f} & Y \ar@{->>}[r]^{j} & Y/Y'
}
        \end{equation}
        where $X/X', Y' \in \mathcal{C}$. We consider the short exact sequence by Noetherian isomorphism:
        \begin{equation}
            0 \to \frac{X' + \ker f}{\ker f} \to \frac{X}{\ker f} \to \frac{X}{X' + \ker f} \to 0.
        \end{equation}
        Note that $\frac{X}{X' + \ker f}$ is a quotient object of $X/X' \in \mathcal{C}$, and
        \begin{equation}
            \frac{X' + \ker f}{\ker f} \simeq \frac{X'}{X' \cap \ker f} \simeq \operatorname{im} (f \circ i) 
        \end{equation}
        is a subobject of $Y' \in \mathcal{C}$. Hence, $\operatorname{im} f \simeq \frac{X}{\ker f}$ lies in $\mathcal{C}$, yielding that $f$ passes through an object in $\mathcal{C}$.
    \end{proof}
\end{lemma}

\begin{remark}\label{rem:stable_quotient}
    Note that a Serre quotient $\mathcal{A} / \mathcal{C}$ is not a stable additive category in general. \cref{lem:zero_morphism_quotient} shows that the stable category is a subcategory with the same objects.
\end{remark}

\subsection{Serre Subcategories}

\begin{notation}
    We denote the localisation sequence of a Serre subcategory by 
    \begin{equation}\label{eq:localisation_sequence}
        \xymatrix{
\mathcal C \ \ar@{>->}[r]^{i} & \mathcal A \ar@{->>}[r]^{Q} & \mathcal A / \mathcal C
        }.
    \end{equation}
    This sequence is defined up to equivalence of categories, rather than isomorphisms.
\end{notation}

\begin{lemma}
    The Serre quotient category $\mathcal{A} / \mathcal{C}$ is Abelian, and $Q$ in \cref{eq:localisation_sequence} is additive (\cref{thm:localisation_fraction}) and exact.
    \begin{proof}
        For any left exact sequence $0 \to K \xrightarrow{i} X \xrightarrow{f} Y$ in $\mathcal{A}$, we claim $0 \to QK \xrightarrow{Qi} QX \xrightarrow{Qf} QY$ is also left exact in $\mathcal{A} / \mathcal{C}$. It suffices to show the universal property of kernels, i.e.,
        \begin{equation}\label{eq:universal_property_kernel}
            0 \to (W, QK) \xrightarrow{(W,Qi)} (W, QX) \xrightarrow{(W,Qf)} (W,QY)\qquad (\forall W \in \mathcal{A}).
        \end{equation}
        By \cref{eq:right_fraction} and exactness of filtrant colimits in $\mathbf{Ab}$, \cref{eq:universal_property_kernel} holds. Dually, $\mathcal{A} / \mathcal{C}$ has cokernels.
        
        It shows that $Q$ preserves kernels and cokernels, hence $\mathcal{A} / \mathcal{C}$ is Abelian and $Q$ is exact.
    \end{proof}
\end{lemma}

\begin{remark}
    Let $\mathcal{C} \rightarrowtail \mathcal{A}$ be Serre subcategory. We show the differences of stable additive quotient $(\mathrm{St}_\mathcal{A})$ and Serre quotient via universal property:
    \begin{enumerate}
        \item for any additive category $\mathcal{D}$ and additive functor $F : \mathcal{A} \to \mathcal{D}$ that vanishes on $\mathcal{C}$, there is a unique additive functor $\widetilde F : \mathrm{St}_\mathcal{C}{\mathcal{A}} \to \mathcal{D}$ such that $F = \widetilde F \circ \pi$ (by universal property \cref{eq:localisation});
        \item for any Abelian category $\mathcal{B}$ and exact functor $F : \mathcal{A} \to \mathcal{B}$ that vanishes on $\mathcal{C}$, there is a unique exact functor $\widetilde F : \mathcal{A}/\mathcal{C} \to \mathcal{B}$ such that $F = \widetilde F \circ Q$ (\cref{cor:universal_property_quotient}).
    \end{enumerate}
    Let $\mathcal{A}$ be Frobenius hereditary Abelian, and $\mathcal{P}$ be the full subcategory of all projective-injective objects. The stable category $\mathrm{St}_\mathcal{P}{\mathcal{A}}$ is triangulated, but not semi-simple in practice. When $\mathcal{A} / \mathcal{C}$ coincides with $\mathrm{St}_\mathcal{C}{\mathcal{A}}$ as categories, $\mathcal{A} / \mathcal{C}$ is Abelian, thus is semi-simple triangulated.
\end{remark}

\begin{example}
    We show that the category of (small) Abelian categories exhibits a group category-analogue. Abelian categories and exact functors are analogous to groups and group homomorphisms. A Serre subcategory is analogous to a normal subgroup. The Serre quotient is analogous to the quotient group.
\end{example}

\begin{lemma}
    The kernel of an exact functor between Abelian categories is a Serre subcategory.
    \begin{proof}
        It is clear by the definition of Serre subcategories.
    \end{proof}
\end{lemma}

\begin{theorem}\label{thm:localisation_functor}
    For any Abelian category $\mathcal{B}$, the localisation sequence yields an inclusion $\mathrm{Fun}_{\mathcal{C} \to 0}(\mathcal{A}, \mathcal{B}) \rightarrowtail \mathrm{Fun}(\mathcal{A} / \mathcal{C}, \mathcal{B})$, which restricted to an isomorphism of categories of exact functors:
    \begin{equation}\label{eq:localisation_functor}
        \xymatrix{
\mathrm{Fun}_{\mathcal C \to 0}(\mathcal A, \mathcal B) \ar@{>->}[rr] &  & \mathrm{Fun}_{S}(\mathcal A, \mathcal B) \ar@{->>}[rr]^{Q^\ast} &  & \mathrm{Fun}(\mathcal A / \mathcal C, \mathcal B) \\
\mathrm{Ex}_{\mathcal C \to 0}(\mathcal A, \mathcal B) \ar@{->}[rr]_{\cong} \ar@{>->}[u] &  & \mathrm{Ex}_{S}(\mathcal A, \mathcal B) \ar@{->}[rr]_{\cong} \ar@{>->}[u] \ar@{->}[rr]^{Q^\ast|_{\mathrm{Ex}}} &  & \mathrm{Ex}(\mathcal A / \mathcal C, \mathcal B) \ar@{>->}[u]
        }.
    \end{equation}
    Here $\mathrm{Fun}_{\mathcal{C} \to 0} \subseteq \mathrm{Fun}$ is the full subcategory consisting of functors that vanish on $\mathcal{C}$.
    \begin{proof}
        Any functor eliminating $\mathcal{C}$ maps $S$ to isomorphisms, yielding the inclusion $\mathrm{Fun}_{\mathcal{C} \to 0}(\mathcal{A}, \mathcal{B}) \rightarrowtail \mathrm{Fun}_S(\mathcal{A}, \mathcal{B})$. The inclusion $\mathrm{Ex}_{\mathcal C \to 0}(\mathcal A, \mathcal B) \rightarrowtail \mathrm{Ex}_{S}(\mathcal A, \mathcal B)$ is clear, making the left square commutative. Clearly, an exact functor $\mathcal{A} \to \mathcal{B}$ maps $\mathcal{C}$ to $0$, if and only if it maps $S$ to isomorphisms. Hence, the left buttom inclusion is an isomorphism. 

        We show $Q^\ast|_{\mathrm{Ex}}$ maps to $\mathrm{Ex}(\mathcal{A}/\mathcal{C},\mathcal{B})$. For any $QA \xrightarrow{Qf} QB\xrightarrow{Qg} QC$ in $\mathcal{A}/\mathcal{C}$ exact at $QB$, we see that $\operatorname{im} f \rightarrowtail B \twoheadrightarrow \operatorname{im}g$ is short exact in $\mathcal{A} / \mathcal{C}$. The induced epimorphism $\frac{B}{\operatorname{im} f} \twoheadrightarrow \operatorname{im} g$ is in $S$. Hence, for any exact functor $F : \mathcal{A} \to \mathcal{B}$, we have the short exact sequence $F(\operatorname{im} f) \rightarrowtail FB \twoheadrightarrow F(\frac{B}{\operatorname{im}f}) \simeq F(\operatorname{im} g)$. Therefore, $FA \xrightarrow{Ff} FB\xrightarrow{Fg} FC$ is exact at $FB$.

        $Q^\ast|_{\mathrm{Ex}}$ is essentially surjective since $Q$ is exact. Note that $Q^\ast$ is an isomorphism of categories. Hence, $Q^\ast|_{\mathrm{Ex}}$ is also an isomorphism of categories. 
    \end{proof}
\end{theorem}

\begin{corollary}\label{cor:universal_property_quotient}
    We conclude a universal property by \cref{eq:localisation_functor}: any exact functor $F : \mathcal{A} \to \mathcal{B}$ that vanishes on $\mathcal{C}$ factors through a unique exact functor $\widetilde F : \mathcal{A}/\mathcal{C} \to \mathcal{B}$ such that $F = \widetilde F \circ Q$.
\end{corollary}

\begin{corollary}
    If any only if the stable category $\mathrm{St}_\mathcal{C}{\mathcal{A}}$ is Abelian, and the quotient $\mathcal{A} \twoheadrightarrow \mathrm{St}_\mathcal{C}{\mathcal{A}}$ is exact, the inclusion (\cref{rem:stable_quotient}) $\mathrm{St}_\mathcal{C} \mathcal{A} \rightarrowtail \mathcal{A} / \mathcal{C}$ is identical.
\end{corollary}

\begin{example}
    We decompose an exact functor $F : \mathcal{A} \to \mathcal{B}$ between Abelian categories as follows:
    \begin{equation}
\xymatrix{
\ker F \ar@{>->}[r] & \mathcal A \ar@{->}[rrr]^{F} \ar@{->>}[rd] \ar@{->}[ld] &  &  & \mathcal B \ar@{->>}[r] & \frac{\mathcal B}{\langle\mathcal A/ \ker F\rangle} \\
\mathrm{St}_{\ker F}(\mathcal A) \ar@/_1pc/@{->}[rrrrr]_{\langle\rangle} \ar@{->}[rr] &  & \mathcal A/ \ker F \ar@{->}[r]^{\langle\rangle} & \langle\mathcal A/ \ker F\rangle \ar@{>->}[ru] \ar@{-->}[rr]^{\simeq} &  & \langle \mathrm{St}_{\ker F}(\mathcal A)\rangle  \ar@{>->}[lu]
}.
    \end{equation}
    Here $\mathrm{St}_{\ker F}(\mathcal{A})$ is the coimage ($\operatorname{im}^1$) of $F$, and $\langle -\rangle$ is the Serre closure of the essential image ($\operatorname{im}^2$). Note that $\langle \mathrm{St}_{\ker F}(\mathcal{A})\rangle$ is a subcatetory of $\mathcal{A} / \ker F$ (\cref{rem:stable_quotient}), it has the same Serre closure as $\mathcal{A} / \ker F$.
\end{example}

We generalise \cref{cor:efface_abelian} as follows.

\begin{proposition}
    Let $\mathcal{B} \to \mathcal{A}$ be an exact inclusion of Abelian categories, $\mathcal{C} \rightarrowtail A$ be a Serre subcategory. Then $\mathcal{B} \cap \mathcal{C} \rightarrowtail \mathcal{B}$ is also a Serre subcategory, and there is an exact functor $\mathcal{B}/(\mathcal{B} \cap \mathcal{C}) \to \mathcal{A}/\mathcal{C}$ which is an inclusion on objects.
    \begin{proof}
        Within $\mathcal{B}$, any subobject of $X \in \mathcal{B} \cap \mathcal{C}$ lies in both $\mathcal{B}$ and $\mathcal{C}$, and so do quotient objects. The extension term in $\mathcal{B} \cap \mathcal{C}$ is also an extension term in $\mathcal{A}$, by exact inclusion. Hence, $\mathcal{B} \cap \mathcal{C}$ is a Serre subcategory of $\mathcal{B}$. The universal property yields the exact functor $\mathcal{B}/(\mathcal{B} \cap \mathcal{C}) \to \mathcal{A}/\mathcal{C}$ (\cref{cor:universal_property_quotient}).
    \end{proof}
\end{proposition}

\begin{remark}
    It is hard to determine whether $\mathcal{B}/(\mathcal{B} \cap \mathcal{C}) \to \mathcal{A}/\mathcal{C}$ is faithful in general. Let $f$ be a morphism in $\mathcal{B}$ factors through an object in $\mathcal{C}$, then $f$ may not factor through an object in $\mathcal{B} \cap \mathcal{C}$.
\end{remark}

\begin{theorem}[Noetherian isomorphism]
    Let $\mathcal{C} \rightarrowtail \mathcal{B}$ and $\mathcal{B} \rightarrowtail \mathcal{A}$ be Serre subcategories. $\mathcal{C} \rightarrowtail \mathcal{A}$ is clearly a Serre subcategory. There is an induced localisation sequence
    \begin{equation}
\xymatrix{
\mathcal C \ar@{=}[d] \ar@{>->}[r]^{I} & \mathcal B \ar@{->>}[r]^{Q} \ar@{>->}[d]^{J} & \mathcal B/\mathcal C \ar@{>-->}[d]^{L} \\
\mathcal C \ar@{>->}[r]^{JI} & \mathcal A \ar@{->>}[r]^{R} \ar@{->>}[d]^{S} & \mathcal A/\mathcal C \ar@{-->>}[d]^{P} \\
 & \mathcal A/\mathcal B \ar@{->}[r]^{\cong} & \frac{\mathcal A/\mathcal C}{\mathcal B/\mathcal C}
}.
    \end{equation}
    \begin{proof}
        We firstly show that $L$ induces a Serre subcategory. To see $L$ is fully faithful, we take any $X,Y \in \mathcal{B}$. The colimit \cref{eq:serre_quotient_colimit} takes the same value in $\mathcal{B}/\mathcal{C}$ and $\mathcal{A}/\mathcal{C}$, since $\mathcal{B} \rightarrowtail \mathcal{C}$ is Serre. To see $\mathcal{B} / \mathcal{C} \rightarrowtail \mathcal{A} / \mathcal{C}$ is Serre, we take any sequence $RA \to RB \to RC$ in $\mathcal{A} / \mathcal{C}$ which is exact at $RB$. Note that this is an image of some exact sequence in $\mathcal{A}$ (see the proof of \cref{thm:localisation_functor}), denoted by $A \to B \to C$. When $RA, RC \in \mathcal{B} / \mathcal{C}$, we assume $A$ and $B$ lie in $\mathcal{B}$ as $L$ is fully faithful. Hence, $B \in \mathcal{B}$ since $\mathcal{B} \rightarrowtail \mathcal{A}$ is Serre. Therefore, $RB \in \mathcal{B} / \mathcal{C}$, completing the proof that $L$ is a Serre subcategory.

        There is a unique exact functor $F: \mathcal{A} / \mathcal{B} \to \frac{\mathcal{A} / \mathcal{C}}{\mathcal{A} / \mathcal{B}}$ by \cref{cor:universal_property_quotient}. The functor $S$ factors through $R$ by $S = TR$, by universal property of localisation. $T$ is also an exact functor by the same verification for $L$. Hence, there is an unique exact functor $G : \frac{\mathcal{A} / \mathcal{C}}{\mathcal{B} / \mathcal{C}} \to \mathcal{A} / \mathcal{B}$ such that $T = G P$. We show $F$ and $G$ are quasi-inverses. The uniqueness shows that $GF$ and $FG$ are identities. 
    \end{proof}
\end{theorem}



\subsection{Localisation Sequences}

Now we consider the Serre quotient of $\mathrm{eff}_\mathcal{A} \rightarrowtail \mathrm{mod}_\mathcal{A}$ in \cref{cor:efface_abelian} and $\mathrm{eff}^a_\mathcal{A} \rightarrowtail \mathrm{mod}_\mathcal{A}$ in \cref{cor:defect_mod}.

\begin{theorem}
    Let $\mathcal{A}$ be additive with weak kernels (under the assumption of \cref{cor:efface_abelian}). Then the there is a localisation sequence of Serre subcategory:
    \begin{equation}\label{eq:localisation_efface}
\xymatrix{
\mathrm{eff}_\mathcal A \ar@{>->}[rr]^{I} &  & \mathrm{mod}_\mathcal A \ar@{->>}[rr]^{Q} &  & \mathcal A \ar@/^1pc/@{>->}[ll]^{\mathbb Y}
}.
    \end{equation}
    Here $I$ is the inclusion, $Q$ is the exact left adjoint of the Yoneda embedding $\mathbb Y : \mathcal{A} \to \mathrm{mod}_\mathcal{A}$. The existence of $Q$ is due to \cref{prop:fp_left_adj}, the exactness is due to \cref{cor:fp_left_adj}.
    \begin{proof}
        The universal property of Serre quotient (\cref{cor:universal_property_quotient}) yields a factorisation
        \begin{equation}
            Q = [\mathrm{mod}_\mathcal{A} \xrightarrow{Q_0} ({\mathrm{mod}_\mathcal{A}} / {\mathrm{eff}_\mathcal{A}}) \xrightarrow{\overline Q} \mathcal{A}].
        \end{equation}
        It suffices to show $\overline Q$ is an equivalence with quasi-inverse $Q_0 \circ \mathbb Y$. The composition $\overline Q \circ (Q_0 \circ \mathbb Y) = Q \circ \mathbb Y \simeq \mathrm{id}_\mathcal{A}$ is clearly an auto-equivalence. Conversely, we show $(Q_0 \circ \mathbb Y) \circ \overline Q \simeq \mathrm{id}_{{\mathrm{mod}_\mathcal{A}} / {\mathrm{eff}_\mathcal{A}}}$. Note that $(Q_0)^\ast$ induces an isomorphism of categories (\cref{def:localisation}), it suffices to find a natural isomorphism $Q_0 \circ \mathbb Y \circ Q \xrightarrow{\simeq} Q_0$. We claim $Q_0 \eta : Q_0 \to Q_0 \circ \mathbb Y \circ Q$ is an natural isomorphism by showing that each $\eta _F : F \to \mathbb Y(Q(F))$ has effaceable kernel and cokernel. We take $F = \operatorname{cok} (-, f)$ for the right exact sequence $X \xrightarrow{f} Y \xrightarrow{p} \operatorname{cok} f \to 0$, then there is a $4$-term exact sequence
        \begin{equation}
            0 \to H \rightarrowtail F \xrightarrow{\eta_F} (-, \operatorname{cok} f) \twoheadrightarrow \operatorname{cok} (-, p) \to 0.
        \end{equation}
        yielding by
        \begin{equation}
\xymatrix{
(-,X) \ar@{->}[rr]^{{(-,f)}} \ar@{->>}[rd] &  & (-,Y) \ar@{->}[rr]^{{(-,p)}} \ar@{->>}[rd] &  & (-, \operatorname {cok}f) \\
 & \operatorname {im}(-,f) \ar@{>->}[ru] \ar@{->>}[rd] &  & F \ar@{->}[ru]_{\eta_F} &  \\
 &  & H \ar@{>->}[ru] &  & 
}.
        \end{equation}
        Since $\operatorname{cok} (-, p)$ is effaceable, and $H$ is a quotient of, we complete the proof.
    \end{proof}
\end{theorem}

\begin{theorem}
    The localisation sequence extend to
    \begin{equation}
        \xymatrix{
\mathrm{eff}_\mathcal A \ar@{>->}[rr]^{I} &  & \mathrm{mod}_\mathcal A \ar@{->>}[rr]^{Q} \ar@/^1pc/@{->>}[ll]^{\mathbb C} &  & \mathcal A \ar@/^1pc/@{>->}[ll]^{\mathbb Y}
        }.
    \end{equation}
    Here $\mathbb C : \operatorname{cok} (-, f) \mapsto \operatorname{cok} (-, p)$ for any $X \xrightarrow f Y \overset p \twoheadrightarrow \operatorname{cok}f$.
    \begin{proof}
        Note that for any epimorphism $q : A \twoheadrightarrow B$,
        \begin{equation}
            [\operatorname{cok}(-, q), \operatorname{cok} (-, f)] \simeq \ker \Big([(-, B), \operatorname{cok} (-, f)] \to [(-, A), \operatorname{cok} (-, f)]\Big) \simeq \ker (\operatorname{cok} (B,f) \to \operatorname{cok} (A,f)).
        \end{equation}
    \end{proof}
\end{theorem}





































\newpage 
\bibliographystyle{alphaurl}
\bibliography{ref}

























\end{document}
